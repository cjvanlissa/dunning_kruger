\section{Definitions}
Define random variables
\begin{itemize}
 \item $s^*$ denoting skill
 \item $\epsilon$ denoting measurement error, with $\Exp[\epsilon] = 0$, $\epsilon$ independent of all other random variables included in the model
 \item $s^*_s$ denoting self-assessed skill
\end{itemize}

\noindent Then we define performance $p$ as
\begin{equation} \label{p}
  p \coloneq s^* + \epsilon
\end{equation}
and overconfidence $oc^*$ as
\begin{equation} \label{oc}
  oc^* \coloneq s^*_s-s^*
\end{equation}
and expected performance $p_e$ as
\begin{equation} \label{ep}
  p_e \coloneq s^* + oc^*
\end{equation}
Overconfidence $oc^*$ is measured by overestimation $oe$ defined as
\begin{equation}
  oe \coloneq p_e - p
\end{equation}

\section{Theorems}

Theorem 1:

\begin{equation}
  oe = oc^* - \epsilon
\end{equation}

Proof 1:
\noindent From eq. \ref{oc} and \ref{ep} it follows that $p_e = s^*_s$ and further from eq. \ref{ep} and \ref{p} we see
\begin{align} \label{dd}
  oe &= p_e - p \\
  &= (s^* + oc^*) - (s^* + \epsilon) \\
  &= oc^* - \epsilon
\end{align}
